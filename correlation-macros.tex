
%%Macros
% This is the definition of meet_r
\def\meetR{%
  \alignedmathop{%
    \bigwedge%
  }{%
    \mathchoice%
    {\bigwedge\mkern-0.5mu\smash{\raisebox{-0.9ex}{$\resetstyle\mathlarger{\mathlarger{{}_{\mathcal{R}}}}$}}}%
    {\bigwedge\mkern-1.5mu\smash{\raisebox{-0.6ex}{$\resetstyle\mathlarger{\mathlarger{{}_{\mathcal{R}}}}$}}}%
    {\bigwedge\mkern-2.25mu\smash{\raisebox{-0.3ex}{$\resetstyle\mathlarger{\mathlarger{{}_{\mathcal{R}}}}$}}}%
    {\bigwedge\mkern-2.5mu\smash{\raisebox{-0.35ex}{$\resetstyle\mathlarger{{}_{\mathcal{R}}}$}}}%
  }%
}
%
% This is the definition of join_k
\def\ijoin{%
  \alignedmathop{%
    \bigvee%
  }{%
    \mathchoice%
    {\bigvee\mkern-6mu\smash{\raisebox{-0.9ex}{$\resetstyle\mathlarger{\mathlarger{{}_{\mathcal{K}}}}$}}}%
    {\bigvee\mkern-5mu\smash{\raisebox{-0.6ex}{$\resetstyle\mathlarger{\mathlarger{{}_{\mathcal{K}}}}$}}}%
    {\bigvee\mkern-2.25mu\smash{\raisebox{-0.3ex}{$\resetstyle\mathlarger{\mathlarger{{}_{\mathcal{K}}}}$}}}%
    {\bigvee\mkern-2.5mu\smash{\raisebox{-0.35ex}{$\resetstyle\mathlarger{{}_{\mathcal{K}}}$}}}%
  }%
}
\def\largeijoin{%
  \alignedmathop{%
    \bigvee%
  }{%
    \mathchoice%
    {\bigvee\mkern-7mu\smash{\raisebox{-0.6ex}{$\resetstyle\mathlarger{{}_{\mathcal{K}}}$}}}%
    {\bigvee\mkern-5mu\smash{\raisebox{-0.6ex}{$\resetstyle\mathlarger{\mathlarger{{}_{\mathcal{K}}}}$}}}%
    {\bigvee\mkern-2.25mu\smash{\raisebox{-0.3ex}{$\resetstyle\mathlarger{\mathlarger{{}_{\mathcal{K}}}}$}}}%
    {\bigvee\mkern-2.5mu\smash{\raisebox{-0.35ex}{$\resetstyle\mathlarger{{}_{\mathcal{K}}}$}}}%
  }%
}



%%For Tikz figures
%%%%% Custom pattern definition (not my code):
% defining the new dimensions
  \newlength{\hatchspread}
  \newlength{\hatchoffset}
  \newlength{\hatchthickness}
  % declaring the keys in tikz
  \tikzset{
    hatchspread/.code={\setlength{\hatchspread}{#1}},
    hatchoffset/.code={\setlength{\hatchoffset}{#1}},
    hatchthickness/.code={\setlength{\hatchthickness}{#1}}
  }
  % setting the default values
  \tikzset{
    hatchspread=.5cm,
    hatchoffset=0.4pt,
    hatchthickness=0.4pt
  }
  % declaring the pattern
  \pgfdeclarepatternformonly[\hatchspread,\hatchthickness]% variables
     {custom north east lines}% name
     {\pgfqpoint{-2\hatchthickness}{-2\hatchthickness}}% lower left corner
     {\pgfqpoint{\dimexpr\hatchspread+2\hatchthickness}{\dimexpr\hatchspread+2\hatchthickness}}% upper right corner
     {\pgfqpoint{\hatchspread}{\hatchspread}}% tile size
     {% shape description
      \pgfsetlinewidth{\hatchthickness}
      \pgfpathmoveto{\pgfqpoint{0pt}{\hatchoffset}}
      \pgfpathlineto{\pgfqpoint{\dimexpr\hatchspread-\hatchoffset}{\hatchspread}}
      \pgfpathmoveto{\pgfqpoint{\dimexpr\hatchspread-\hatchoffset}{0pt}}
      \pgfpathlineto{\pgfqpoint{\dimexpr\hatchspread}{\hatchoffset}}
      \pgfusepath{stroke}
     }
%%%%% End custom pattern definition.

%\pgfmathparse{\year+\time+\currenthour+\currentminute*\currentsecond}
\edef\rndseed{4101} %%%%% To choose a fixed seed
%\edef\rndseed{\pgfmathresult} %%%%% Random seed
\pgfmathsetseed{\rndseed}

\def\cellw{.7cm}
\def\cellh{.8cm}
\def\minth{.15cm}
\def\vtpadding{.2cm}
\def\vbpadding{.1cm}
\def\hpadding{.05cm}
\def\nbnodes{4}
\def\minrand{0.3}
\def\borderh{0.2cm}

\def\triangleArray#1#2{
  \foreach \i in {0,...,\nbnodes} {%
    \node[inner sep=0pt,minimum height=\cellh, minimum width=\cellw, alias=#1-last, anchor=north west, xshift=\i*\cellw] (#1-\i) at #2 {};
    
    % Magic coordinate calculations
    % Fixed top position:
    \pgfmathsetmacro{\tpos}{0}
    % Random top position:
    % \pgfmathsetmacro{\tpos}{rnd*(1-\minrand)+\minrand}
    \pgfmathsetmacro{\trih}{rnd*(1-\minrand)+\minrand}
    \def\tposformula{(\vtpadding+\tpos*(\cellh-\minth-\vtpadding-\vbpadding))}
    \def\thformula{min(\trih*(\cellh-\tposformula-\vbpadding), \cellw*(1/.5555)*.5-2*\hpadding)}
    % Triangle:
    \draw #2 ++(\i*\cellw+.5*\cellw,{-\tposformula}) coordinate (#1-tt-\i) -- ++({.5555*\thformula},{-\thformula}) coordinate (#1-tr-\i) -- ++({-.5555*\thformula*2},0) coordinate (#1-tl-\i) -- cycle;
    \coordinate (#1-tc-\i) at (barycentric cs:#1-tt-\i=1,#1-tr-\i=1,#1-tl-\i=1);
    \coordinate (#1-tb-\i) at ($(#1-tl-\i)!.5!(#1-tr-\i)$);
  }
  
  % \draw (#1-0.north west) ++(0,\borderh) -- ($(#1-last.north east)+(0,\borderh)$) -- (#1-last.south east) -- (#1-0.south west) -- cycle;
  % \draw (#1-0.north west) -- (#1-last.north east);

  \draw (#1-last.north east) -- (#1-last.south east) -- (#1-0.south west) -- (#1-0.north west);
  \draw[fill=black!7] (#1-0.north west) -- (#1-last.north east) -- ++(0,\borderh) -- ($(#1-0.north west)+(0,\borderh)$) -- cycle;
  
  \foreach \i in {1,...,\nbnodes} {
    \draw (#1-\i.north west) -- (#1-\i.south west);
  }
}

%Leq for pequiv
\def\rleq{\mathrel{\sqsubseteq_{\kern0.15pt\rdom}}}
\def\rjoin{\mathrel{\vee_{\kern-0.25pt\rdom}}}
\def\rmeet{\mathrel{\wedge_{\kern0.15pt\rdom}}}
%\DeclareMathOperator*{\rbigwedge}{\bigwedge_\rdom}


%Macros for pequiv relations
\def\rdom{\mathscr{R}}
\def\equal{\textsf{Equal}}
\def\qmark{\textsf{Any}}

%\def\rel{\mathit{\mathcal{R}}}
\def\rel{R}
\newcommand{\srel}[2]{\{#1_1 \mapsto \rel_1; \ldots; #1_{#2} \mapsto \rel_{#2}\}}
\newcommand{\sreljoin}[2]{\{#1_1 \mapsto \rel_1 \rjoin \rel'_1; \ldots; #1_{#2} \mapsto \rel_{#2} \rjoin \rel'_{#2} \}}
\newcommand{\srelmeet}[2]{\{#1_1 \mapsto \rel_1 \rmeet \rel'_1; \ldots; #1_{#2} \mapsto \rel_{#2} \rmeet \rel'_{#2} \}}
\newcommand{\sreli}[2]{\{#1_1 \mapsto \rel_1; \ldots; #1_i \mapsto \rel_i; \ldots; #1_{#2} \mapsto \rel_{#2}\}}
\newcommand{\srelitop}[3]{\{#1_1 \mapsto \qmark; \ldots; #1_i \mapsto #3; \ldots; #1_{#2} \mapsto \qmark\}}
\def\srelibot{\{f_1 \mapsto \equal; \ldots; f_i \mapsto \qmark; \ldots; f_n \mapsto \equal\}}

\newcommand{\vrel}[2]{[#1_1 \mapsto \rel_1; \ldots; #1_{#2} \mapsto \rel_{#2}]}
\newcommand{\vreljoin}[2]{[#1_1 \mapsto \rel_1 \rjoin \rel'_1; \ldots; #1_{#2} \mapsto \rel_{#2} \rjoin \rel'_{#2}]}
\newcommand{\vrelmeet}[2]{[#1_1 \mapsto \rel_1 \rmeet \rel'_1; \ldots; #1_{#2} \mapsto \rel_{#2} \rmeet \rel'_{#2}]}
\newcommand{\vreli}[2]{[#1_1 \mapsto \rel_1; \ldots; #1_i \mapsto \rel_i; \ldots; #1_{#2} \mapsto \rel_{#2}]}
\newcommand{\vrelitop}[3]{[#1_1 \mapsto \qmark; \ldots; #1_i \mapsto #3; \ldots; #1_{#2} \mapsto \qmark]}


\newcommand{\arel}[1]{\langle #1 \rangle} 
\newcommand{\aerel}[3]{\left\langle #1 \triangleright #2: \; #3 \right\rangle}
\newcommand{\aerelnoleftright}[3]{\langle #1 \triangleright #2: \; #3 \rangle}

\newcommand{\rels}{\srel{f}{n}}
\newcommand{\relsi}{\sreli{f}{n}}
\newcommand{\relv}{\vrel{C}{n}}
\newcommand{\relvi}{\vreli{C}{n}}
\newcommand{\rela}{\arel{\rel}}
\newcommand{\relai}{\langle \rel_{\mathit{def}} \triangleright i \;: \; \rel_{\mathit{exc}} \rangle}
\def\relaitop{\langle \equal \triangleright i \;: \; \qmark \rangle}

\newcommand{\srelv}[3]{\{#1_1 \mapsto #3_1; \ldots; #1_{#2} \mapsto #3_{#2}\}}
\newcommand{\vrelv}[3]{[#1_1 \mapsto #3_1; \ldots; #1_{#2} \mapsto #3_{#2}]}

\newcommand{\semrel}[2]{\llbracket #1 \rrbracket^{#2}}


%Projections on pequiv
\def\fieldp{\textrm{\textsf{\emph{extr}}}_f}
\def\consp{\textrm{\textsf{\emph{extr}}}_C}
\def\allp{\textrm{\textsf{\emph{extr}}}_{\langle * \rangle}}
\def\cellp{\textrm{\textsf{\emph{extr}}}_{\langle i \rangle}}
\def\outp{\textrm{\textsf{\emph{extr}}}_{\langle * \setminus i \rangle}}


%Macros for paths
\def\pcell{\langle i \rangle}
\def\pstruct{.f}
\def\pvar{@C}
\def\pempty{\widehat{\varepsilon}}
\def\identical{\textsf{Identical}}
\def\incompatible{\textsf{Incompatible}}
\def\leftcase{\textsf{Left}}
\def\rightcase{\textsf{Right}}
\def\findl{\curlywedge}
\def\append{\dblcolon}

%Macros for path operators:
%\DeclareMathOperator{\pproj}{\rightsquigarrow}
%\DeclareMathOperator{\pinj}{\curvearrowleft}

%Macros for correlations
%\def\cor{\hat{\mathcal{C}}}
\def\cor{\mathscr{K}}
%\def\corv{\hat{c}}
\def\corv{\kappa}
\def\aleq{\mathrel{\hat{\sqsubseteq}}}
%\DeclareMathOperator*{\ajoin}{\hat{\bigvee}}
%\DeclareMathOperator*{\hatbigwedge}{\hat{\bigwedge}}
\def\acompose{\mathop{\ocircle}}
\def\raligned{\rel_{\aligned(\pi', \rho')}^{(\pi, \rho)}}
\def\ralignedc{\rel_{\aligned\corv}^{(\pi, \rho)}}
\def\ralignedv{\rel_{\aligned\corv_1}^{(\pi, \rho)}}
\def\ralignedw{\rel_{\aligned\corv_2}^{(\pi, \rho)}}

%Macros for intra domains
\def\ileq{\sqsubseteq_{\mathcal{K}}}
%\def\ijoin{\bigvee_{\mathcal{K}}}
\def\justif{\Vvdash}
\def\aligned{\parallel}
\def\intracor{K}
\def\compose{\odot}

%Macros for equations
\newcommand{\Ctransfer}[3]{\mathbb{C}^{#2}_{#3}(#1)}

\def\inter{\mathcal{K}_p}
%Colors
%\definecolor{indianred}{rgb}{0.8, 0.36, 0.36}
\definecolor{iburg}{rgb}{0.75, 0.0, 0.2}

% hat math variables

\def\cptype{\ensuremath{\widehat{\Pi}}}
\def\cpath{\ensuremath{\widehat{\pi}}}
\def\crho{\ensuremath{\widehat{\rho}}}
\def\csigma{\ensuremath{\widehat{\sigma}}}
\def\calpha{\ensuremath{\widehat{\alpha}}}
\def\cbeta{\ensuremath{\widehat{\beta}}}
\def\cgamma{\ensuremath{\widehat{\gamma}}}
\def\cdelta{\ensuremath{\widehat{\delta}}}

\def\kset{$\textrm{\emph{kill}}_\lambda$}
\def\kseti{$\textrm{\emph{kill}}_i$}
\def\ksettrue{$\textrm{\emph{kill}}_{\mathit{true}}$}
\def\dls{c_{\lambda}^s}
